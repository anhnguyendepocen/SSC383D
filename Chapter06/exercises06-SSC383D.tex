  \documentclass{mynotes}

%\geometry{showframe}% for debugging purposes -- displays the margins

\newcommand{\E}{\mbox{E}}
\newcommand{\MSE}{\mbox{MSE}}
\newcommand{\var}{\mbox{var}}


\usepackage{amsmath}
%\usepackage[garamond]{mathdesign}
\usepackage{url}

% Set up the images/graphics package
\usepackage{graphicx}
\setkeys{Gin}{width=\linewidth,totalheight=\textheight,keepaspectratio}
\graphicspath{{graphics/}}

\title[Exercises 6 $\cdot$ SSC 383D]{Exercises 6: Latent-variable models}
%\author[ ]{ }
\date{}  % if the \date{} command is left out, the current date will be used

% The following package makes prettier tables.  We're all about the bling!
\usepackage{booktabs}

% The units package provides nice, non-stacked fractions and better spacing
% for units.
\usepackage{units}

% The fancyvrb package lets us customize the formatting of verbatim
% environments.  We use a slightly smaller font.
\usepackage{fancyvrb}
\fvset{fontsize=\normalsize}

% Small sections of multiple columns
\usepackage{multicol}

% Provides paragraphs of dummy text
\usepackage{lipsum}

% These commands are used to pretty-print LaTeX commands
\newcommand{\doccmd}[1]{\texttt{\textbackslash#1}}% command name -- adds backslash automatically
\newcommand{\docopt}[1]{\ensuremath{\langle}\textrm{\textit{#1}}\ensuremath{\rangle}}% optional command argument
\newcommand{\docarg}[1]{\textrm{\textit{#1}}}% (required) command argument
\newenvironment{docspec}{\begin{quote}\noindent}{\end{quote}}% command specification environment
\newcommand{\docenv}[1]{\textsf{#1}}% environment name
\newcommand{\docpkg}[1]{\texttt{#1}}% package name
\newcommand{\doccls}[1]{\texttt{#1}}% document class name
\newcommand{\docclsopt}[1]{\texttt{#1}}% document class option name

\newcommand{\N}{\mbox{N}}
\newcommand{\thetahat}{\hat{\theta}}
\newcommand{\sigmahat}{\hat{\sigma}}
\newcommand{\betahat}{\hat{\beta}}


\begin{document}

\maketitle% this prints the handout title, author, and date

\section{Latent class models}

\subsection{Multinomial--Dirichlet conjugacy}

The multinomial distribution is the generalization of the binomial to multiple categories.  A vector of counts $(n_1, \ldots, n_K)$ has a multinomial distribution with parameter $w = (w_1, \ldots, w_K)$ if the $w_k$'s sum to 1 and if
$$
p(n_1, \ldots, n_K \mid w) = \frac{N!}{n_1! \cdots n_K!} \ \prod_{k=1}^K w_k^{n_k} \, ,
$$
where $N = \sum_{k=1}^K n_k$.  The normalizing constant can be derived from the same kind of combinatorial argument that applies to the binomial distribution.

\begin{enumerate}
\item Show that the Dirichlet distribution, written $w \sim \mbox{Dir}(\alpha_1, \ldots, \alpha_K)$ and with p.d.f.
$$
p(w \mid \alpha) = \frac{\Gamma(\sum \alpha_k) }{\sum \Gamma(\alpha_k) } \prod_{k=1}^K w_k^{\alpha_k -1} \, ,
$$
is the conjugate prior for the multinomial distribution.  (The support of the Dirichlet distribution is the $K$-dimensional simplex.)
\item Let $x_k \sim \mbox{Ga}(\alpha_k, 1)$ be independent Gamma draws for $k=1, \ldots, K$.  Show that the random vector with elements $w_k = x_k/X$, where $X = \sum{x_k}$, has a Dirichlet distribution with parameter $(\alpha_1, \ldots, \alpha_K)$.  (Recall that you showed a very similar thing for the beta distribution; the Dirichlet is easily understood as the multiple-category generalization of the beta.)
\end{enumerate}

\subsection{A simple mixture model}

Imagine a high school where every sophomore takes a standardized math test.  This school has three kinds of sophomore math classes:
\begin{description}
\item[remedial classes,] with a cumulative enrollment of about 100 students.  Among students in the remedial classes, the test scores are normally distributed with mean 55 and standard deviation 15.
\item[average classes,] cumulative enrollment 400 students.  Among these students, the test scores are normally distributed with mean 70 and standard deviation 10.
\item[honors classes,] cumulative enrollment 150. Among these students, the test scores are normally distributed with mean 85 and standard deviation 5.
\end{description}

The following problems all concern this stylized scenario.
\begin{enumerate}
\item Simulate a few different data sets according to this scenario.  How would you describe the school-wide distribution of test scores? (Skewed, multi-modal, etc.)
\item Let $\gamma_i$ denote a latent indicator variable for student $i$, that takes the value $k$ if student $i$ falls in class $k$.  Let $w = (w_1, w_2, w_3)$ denote the prior probabilities of the three classes.  Write functions to draw from the conditional posterior distributions for: (1) $\gamma_i$, given $w$ and the information about the distribution of students within classes; and (2) $w$, given all the $\gamma_i$'s.
\item Suppose you allocate every student into one of three classes by drawing from the conditional distribution of $\gamma_i$.  Now pretend you didn't know the true mean or variance of the scores within each class, and that you merely had (sensibly chosen) priors for them.  What would the posterior distribution be for the mean and variance of students within bucket $k$?\footnote[][-3pc]{If you want to superimpose a plot of the three densities on top of the histogram of test scores, use the ``prob=TRUE'' flag to the histogram function.  Then you can add lines to the plot.}
\end{enumerate}
Propose a general Gibbs sampler for the following problem: the marginal density of some outcome $Y$ is a $K$-component mixture of normals, but you don't know the probabilities, means, or variances of each of the components.  Try running your Gibbs sampler on one of the simulated test-score data sets, and see how well it does at recovering the true density.  (Notice how you get error bars for free from your Gibbs sampler.) 

Then, use your Gibbs sampler to fit a mixture of normals to the data in ``galaxies.txt,'' which shows the recessional velocities for 82 different galaxies (i.e.~how fast those galaxies are moving away from us) from 6 different well-separated conic sections of space.  Our current understanding of galaxy formation would predict just such a mixture distribution, with each cluster corresponding to a ``clump'' of galaxies, and each galaxy within a cluster moving at a similar speed.



\section{Latent feature models}

\end{document}

